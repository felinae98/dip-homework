\documentclass[UTF8]{ctexart}
\usepackage{minted}
\title{Visual Odomentry代码阅读报告}
\author{唐誉铭}
\date{\today}
\begin{document}
	\maketitle
	\section{论文概要}
	\subsection{论文要解决的问题}
	论文中提出了一种基于双镜摄像头拍摄的视频进行三维重建的方法。3D感知是计算机视觉和机器人技术的核心课题之一。在实践中,相机的分辨率严重受限,并且对生成结果的准确性有着较高的要求,
	\section{代码分析}
    \subsection{整体流程分析}
    \subsubsection{demo程序分析}
    \begin{minted}[linenos, breaklines, breakanywhere, mathescape]{c++}
VisualOdometryStereo::parameters param;

// calibration parameters for sequence 2010_03_09_drive_0019 
param.calib.f  = 645.24; // focal length in pixels
param.calib.cu = 635.96; // principal point (u-coordinate) in pixels
param.calib.cv = 194.13; // principal point (v-coordinate) in pixels
param.base     = 0.5707; // baseline in meters
//新建一个参数,并且设置参数
// init visual odometry
VisualOdometryStereo viso(param);
//通过参数初始化一个VisualOdometryStereo实例
    \end{minted}
    新建参数并且初始化一个实例
    \begin{minted}[linenos, breaklines, breakanywhere, mathescape]{c++}
int32_t dims[] = {width,height,width};
if (viso.process(left_img_data,right_img_data,dims)) {

// on success, update current pose
pose = pose * Matrix::inv(viso.getMotion());

// output some statistics
double num_matches = viso.getNumberOfMatches();
double num_inliers = viso.getNumberOfInliers();
cout << ", Matches: " << num_matches;
cout << ", Inliers: " << 100.0*num_inliers/num_matches << " %" << ", Current pose: " << endl;
cout << pose << endl << endl;

} else {
cout << " ... failed!" << endl;
}
    \end{minted}
    分别从从左右镜头读入一帧,把图片中的信息存入数组,调用VisualOdemetryStereo的process方法进行处理,处理完成后输出匹配成功的点,内围点的占比和当前摄像机的姿态
    \subsubsection{主要类分析}
    \paragraph{VisualOdomentry简要分析}
    \begin{minted}[linenos, breaklines, breakanywhere, mathescape]{c++}
class VisualOdometry {
public:
    //相机校准参数
    struct calibration {  
       ...
    };
  
    // bucketing 参数
    struct bucketing {  
      ...
    };
  
    // 参数
    struct parameters {
    };
protected:
    Matrix                         Tr_delta;   // 姿态变换
    bool                           Tr_valid;   // 是否存在预测的下一帧
    Matcher                       *matcher;    // feature matcher
    std::vector<int32_t>           inliers;    // 内围点
    double                        *J;          // 雅克比行列式
    double                        *p_observe;  // 观测点的二维坐标
    double                        *p_predict;  // 预测点的二维坐标
    std::vector<Matcher::p_match>  p_matched;  // feature point matches
    
private:
    
    parameters                    param;     // 参数
}
    \end{minted}
    \paragraph{VisualOdometryStereo简要分析}
    \begin{minted}[linenos, breaklines, breakanywhere, mathescape]{c++}
class VisualOdometryStereo {
    % TODO
}
    \end{minted}
    \subsubsection{主要过程分析}
    \begin{minted}[linenos, breaklines, breakanywhere, mathescape]{c++}
bool VisualOdometryStereo::process (uint8_t *I1,uint8_t *I2,int32_t* dims,bool replace) {

    matcher->pushBack(I1,I2,dims,replace);
    // 把左右两帧图片加入ringbuffer,并计算特征值
    
    // 如果是处理第一张图片就先建立动作预测
    if (!Tr_valid) {
        matcher->matchFeatures(2);
        matcher->bucketFeatures(param.bucket.max_features,param.bucket.bucket_width,param.bucket.bucket_height);                          
        p_matched = matcher->getMatches();
        updateMotion();
    }
    
    if (Tr_valid) matcher->matchFeatures(2,&Tr_delta);
    else          matcher->matchFeatures(2);
    // 使用环形匹配法匹配图像特征
    matcher->bucketFeatures(param.bucket.max_features,param.bucket.bucket_width,param.bucket.bucket_height);                          
    // 减少特征数量,并使特征值较为均匀的分布在图像中
    p_matched = matcher->getMatches();
    return updateMotion();
    // 更新动作
}
        
    \end{minted}
    \paragraph{Matcher类分析}
    \begin{minted}[linenos, breaklines, breakanywhere, mathescape]{c++}

    \end{minted}
    \subsection{局部模块分析}
    \subsubsection{功能类分析}
    \subsubsection{功能函数分析}
    \paragraph{Matcher::pushBack} 
        把左右各一帧图像放入ringbuffer,并且计算两张图像的各个特征
    \begin{minted}[linenos, breaklines, breakanywhere, mathescape]{c++}
void Matcher::pushBack (uint8_t *I1,uint8_t* I2,int32_t* dims,const bool replace) {
    
    // 定义图片大小
    int32_t width  = dims[0];
    int32_t height = dims[1];
    int32_t bpl    = dims[2];

    // sanity check
    if (width<=0 || height<=0 || bpl<width || I1==0) {
        cerr << "ERROR: Image dimension mismatch!" << endl;
        return;
    }

    if (replace) {
        ... //如果规定了replace, 释放上上张图片的各项参数
    } else {
        ... //释放上上张图片的各项参数
        ... //将“当前”图片的各项信息设为“上张”图片的各项信息
    }

    // 对齐内存
    dims_c[0] = width;
    dims_c[1] = height;
    dims_c[2] = width + 15-(width-1)%16;

    I1c = (uint8_t*)_mm_malloc(dims_c[2]*dims_c[1]*sizeof(uint8_t),16);
    I2c = (uint8_t*)_mm_malloc(dims_c[2]*dims_c[1]*sizeof(uint8_t),16);
    ... // 对齐放置图片信息

    // 计算图片各项特征
    computeFeatures(I1c,dims_c,m1c1,n1c1,m1c2,n1c2,I1c_du,I1c_dv,I1c_du_full,I1c_dv_full);
    if (I2!=0)
        computeFeatures(I2c,dims_c,m2c1,n2c1,m2c2,n2c2,I2c_du,I2c_dv,I2c_du_full,I2c_dv_full);
    }
    \end{minted}
    \paragraph{Matcher::computeFeatures}
    \begin{minted}[linenos, breaklines, breakanywhere, mathescape]{c++}
void Matcher::computeFeatures (...) {
  
    ...

    if (!param.half_resolution) {
        ... // demo不涉及这个部分
    } else {
        uint8_t* I_matching = createHalfResolutionImage(I,dims);
        getHalfResolutionDimensions(dims,dims_matching);
        // 将图像缩小一半
        ... // 为各个滤波结果分配空间
        filter::sobel5x5(I_matching,I_du,I_dv,dims_matching[2],dims_matching[1]);
        // 对缩略图进行sobel滤波求梯度
        filter::sobel5x5(I,I_du_full,I_dv_full,dims[2],dims[1]);
        // 对原图图进行sobel滤波求梯度
        filter::blob5x5(I_matching,I_f1,dims_matching[2],dims_matching[1]);
        // 对缩略图使用blob滤波
        filter::checkerboard5x5(I_matching,I_f2,dims_matching[2],dims_matching[1]);
        // 对缩略图求边界
        _mm_free(I_matching);
    }
    
    // 使用非极大抑制提取稀疏极大值
    vector<Matcher::maximum> maxima1;
    if (param.multi_stage) {
        int32_t nms_n_sparse = param.nms_n*3;
        if (nms_n_sparse>10)
            nms_n_sparse = max(param.nms_n,10);
        nonMaximumSuppression(I_f1,I_f2,dims_matching,maxima1,nms_n_sparse);
        computeDescriptors(I_du,I_dv,dims_matching[2],maxima1);
    }
    
    // 使用非极大抑制提取稠密极大值
    vector<Matcher::maximum> maxima2;
    nonMaximumSuppression(I_f1,I_f2,dims_matching,maxima2,param.nms_n);
    computeDescriptors(I_du,I_dv,dims_matching[2],maxima2);

    ...

    if (num1!=0) {
        ... // 将稀疏最大值对齐内存存入返回变量
    }
    
    if (num2!=0) {
        ... // 将稠密最大值对齐内存存入返回变量
    }
}

    \end{minted}
    \paragraph{Matcher::matchFeatures}
    
    匹配特征值
    \begin{minted}[linenos, breaklines, breakanywhere, mathescape]{c++}
void Matcher::matchFeatures(int32_t method, Matrix *Tr_delta) {

    ... // 检查完整性

    // double pass matching
    if (param.multi_stage) {

        // 1st pass 稀疏特征匹配
        matching(m1p1,m2p1,m1c1,m2c1,n1p1,n2p1,n1c1,n2c1,p_matched_1,method,false,Tr_delta);
        // 使用2d Delaunay三角剖分消除异常值
        removeOutliers(p_matched_1,method);
        
        // 为加速第二次处理计算搜索范围优先级数据
        computePriorStatistics(p_matched_1,method);      

        // 2nd pass 稠密特征匹配
        matching(m1p2,m2p2,m1c2,m2c2,n1p2,n2p2,n1c2,n2c2,p_matched_2,method,true,Tr_delta);
        if (param.refinement>0)
            // 通过抛物线拟合的子像素细化来进一步改进特征定位
            refinement(p_matched_2,method);
        removeOutliers(p_matched_2,method);

    // single pass matching
    } else {
        ...
    }
}
    \end{minted}
    \paragraph{Matcher::matching}
    \begin{minted}[linenos, breaklines, breakanywhere, mathescape]{c++}
void Matcher::matching (int32_t *m1p,int32_t *m2p,int32_t *m1c,int32_t *m2c,
                        int32_t n1p,int32_t n2p,int32_t n1c,int32_t n2c,
                        vector<Matcher::p_match> &p_matched,int32_t method,bool use_prior,Matrix *Tr_delta) {

    
    
    // loop variables
    int32_t* M = (int32_t*)calloc(dims_c[0]*dims_c[1],sizeof(int32_t));
    int32_t i1p,i2p,i1c,i2c,i1c2,i1p2;
    int32_t u1p,v1p,u2p,v2p,u1c,v1c,u2c,v2c;
    
    double t00,t01,t02,t03,t10,t11,t12,t13,t20,t21,t22,t23;
    if (Tr_delta) {
        ... // 如果存在之前的姿态变化量,就初始化变量
    }

    /////////////////////////////////////////////////////
    // method: flow
    if (method==0) {
        ... // demo不涉及这个部分
    }
        
    /////////////////////////////////////////////////////
    // method: stereo
    } else if (method==1) {
        ... // demo不涉及这个部分
    }
        
    /////////////////////////////////////////////////////
    // method: quad matching
    } else {
        
        // create position/class bin index vectors
        createIndexVector(m1p,n1p,k1p,u_bin_num,v_bin_num);
        createIndexVector(m2p,n2p,k2p,u_bin_num,v_bin_num);
        createIndexVector(m1c,n1c,k1c,u_bin_num,v_bin_num);
        createIndexVector(m2c,n2c,k2c,u_bin_num,v_bin_num);
        
        // for all points do
        for (i1p=0; i1p<n1p; i1p++) {
            // 对所有“前一帧”左边视图的所有特征点执行下面的操作

            // 读取每个特征点的坐标
            u1p = *(m1p+step_size*i1p+0);
            v1p = *(m1p+step_size*i1p+1);

            // compute row and column of statistics bin to which this observation belongs
            int32_t u_bin = min((int32_t)floor((float)u1p/(float)param.match_binsize),u_bin_num-1);
            int32_t v_bin = min((int32_t)floor((float)v1p/(float)param.match_binsize),v_bin_num-1);
            int32_t stat_bin = v_bin*u_bin_num+u_bin;

            // match in circle
            // 开始环形匹配
            findMatch(m1p,i1p,m2p,step_size,k2p,u_bin_num,v_bin_num,stat_bin,i2p, 0,false,use_prior);
            // 匹配“前一帧”左视图和“前一帧”右视图的稀疏特征

            u2p = *(m2p+step_size*i2p+0);
            v2p = *(m2p+step_size*i2p+1);

            if (Tr_delta) {
            
                double d = max((double)u1p-(double)u2p,1.0);
                double x1p = ((double)u1p-param.cu)*param.base/d;
                double y1p = ((double)v1p-param.cv)*param.base/d;
                double z1p = param.f*param.base/d;

                double x2c = t00*x1p + t01*y1p + t02*z1p + t03 - param.base;
                double y2c = t10*x1p + t11*y1p + t12*z1p + t13;
                double z2c = t20*x1p + t21*y1p + t22*z1p + t23;

                double u2c_ = param.f*x2c/z2c+param.cu;
                double v2c_ = param.f*y2c/z2c+param.cv;

                // 如果有之前运动矩阵,就沿用这个运动矩阵
                // 这里假设运动是连续的,运动的变化较小
                findMatch(m2p,i2p,m2c,step_size,k2c,u_bin_num,v_bin_num,stat_bin,i2c, 1,true ,use_prior,u2c_,v2c_);
            } else {
                findMatch(m2p,i2p,m2c,step_size,k2c,u_bin_num,v_bin_num,stat_bin,i2c, 1,true ,use_prior);
            }
            // 匹配“前一帧”右侧图像和“当前帧”右侧图像
            findMatch(m2c,i2c,m1c,step_size,k1c,u_bin_num,v_bin_num,stat_bin,i1c, 2,false,use_prior);
            // 匹配”当前帧“左侧图像和”当前帧“右侧图像
            if (Tr_delta)
                findMatch(m1c,i1c,m1p,step_size,k1p,u_bin_num,v_bin_num,stat_bin,i1p2,3,true ,use_prior,u1p,v1p);
            else
                findMatch(m1c,i1c,m1p,step_size,k1p,u_bin_num,v_bin_num,stat_bin,i1p2,3,true ,use_prior);
            // 匹配”当前帧“左侧图像和”前一帧“右侧图像
            if (i1p2==i1p) {

                u2c = *(m2c+step_size*i2c+0); v2c = *(m2c+step_size*i2c+1);
                u1c = *(m1c+step_size*i1c+0); v1c = *(m1c+step_size*i1c+1);
                if (u1p>=u2p && u1c>=u2c) {
                
                    // 如果匹配成功,就把匹配的结果放入p_match中
                    p_matched.push_back(Matcher::p_match(u1p,v1p,i1p,u2p,v2p,i2p,u1c,v1c,i1c,u2c,v2c,i2c));
                }
            }
        }
    }

    // free memory
    ...
}
    \end{minted}
    \begin{minted}[linenos, breaklines, breakanywhere, mathescape]{c++}
void Matcher::bucketFeatures(int32_t max_features,float bucket_width,float bucket_height) {

    // 找到”当前帧“左侧图像u-v坐标做大的特征点
    float u_max = 0;
    float v_max = 0;
    for (vector<p_match>::iterator it = p_matched_2.begin(); it!=p_matched_2.end(); it++) {
        if (it->u1c>u_max) u_max=it->u1c;
        if (it->v1c>v_max) v_max=it->v1c;
    }

    // 分配需要的buckets
    int32_t bucket_cols = (int32_t)floor(u_max/bucket_width)+1;
    int32_t bucket_rows = (int32_t)floor(v_max/bucket_height)+1;
    vector<p_match> *buckets = new vector<p_match>[bucket_cols*bucket_rows];

    // 把每个特征点放入其位置对应的bukect中
    for (vector<p_match>::iterator it=p_matched_2.begin(); it!=p_matched_2.end(); it++) {
        int32_t u = (int32_t)floor(it->u1c/bucket_width);
        int32_t v = (int32_t)floor(it->v1c/bucket_height);
        buckets[v*bucket_cols+u].push_back(*it);
    }
    
    p_matched_2.clear();
    // 清空原特征点
    for (int32_t i=0; i<bucket_cols*bucket_rows; i++) {
        // 对每个bucket
        
        std::random_shuffle(buckets[i].begin(),buckets[i].end());
        
        // 每个bucket随机填入max_features个特征点
        int32_t k=0;
        for (vector<p_match>::iterator it=buckets[i].begin(); it!=buckets[i].end(); it++) {
            p_matched_2.push_back(*it);
            k++;
            if (k>=max_features)
                break;
        }
    }

    // free buckets
    delete []buckets;
}        
    \end{minted}
    \begin{minted}[linenos, breaklines, breakanywhere, mathescape]{c++}
void Matcher::computePriorStatistics (vector<Matcher::p_match> &p_matched,int32_t method) {
    
    // 计算区域(bin)数量
    int32_t u_bin_num = (int32_t)ceil((float)dims_c[0]/(float)param.match_binsize);
    int32_t v_bin_num = (int32_t)ceil((float)dims_c[1]/(float)param.match_binsize);
    int32_t bin_num   = v_bin_num*u_bin_num;
    
    ...

    for (vector<Matcher::p_match>::iterator it=p_matched.begin(); it!=p_matched.end(); it++) {
        // 对每一组配对好的点组
        // method flow: compute position delta
        if (method==0) {
            ... 
        } else if (method==1) {
            ... 
        } else {
            delta_curr.val[0] = it->u2p - it->u1p;
            delta_curr.val[1] = 0;
            delta_curr.val[2] = it->u2c - it->u2p;
            delta_curr.val[3] = it->v2c - it->v2p;
            delta_curr.val[4] = it->u1c - it->u2c;
            delta_curr.val[5] = 0;
            delta_curr.val[6] = it->u1p - it->u1c;
            delta_curr.val[7] = it->v1p - it->v1c;
            // 计算这个点组的各个偏移量
        }
        
        // 计算哪些区域(bin)包含了这个点组
        int32_t u_bin_min,u_bin_max,v_bin_min,v_bin_max;

        // flow + stereo: use current left image as reference
        if (method<2) {
            ...
        
        } else {
            u_bin_min = min(max((int32_t)floor(it->u1p/(float)param.match_binsize)-1,0),u_bin_num-1);
            u_bin_max = min(max((int32_t)floor(it->u1p/(float)param.match_binsize)+1,0),u_bin_num-1);
            v_bin_min = min(max((int32_t)floor(it->v1p/(float)param.match_binsize)-1,0),v_bin_num-1);
            v_bin_max = min(max((int32_t)floor(it->v1p/(float)param.match_binsize)+1,0),v_bin_num-1);
        }
        
        // 在相关区域对应的accumulator中加入计算出的偏移量
        for (int32_t v_bin=v_bin_min; v_bin<=v_bin_max; v_bin++)
            for (int32_t u_bin=u_bin_min; u_bin<=u_bin_max; u_bin++)
                delta_accu[v_bin*u_bin_num+u_bin].push_back(delta_curr);
    }
    
    ranges.clear();
    
    // 对每个区域(bin)计算最大偏移量
    for (int32_t v_bin=0; v_bin<v_bin_num; v_bin++) {
        for (int32_t u_bin=0; u_bin<u_bin_num; u_bin++) {
            
            // 如果此区域(bin)没有相关记录,就使用默认值
            delta delta_min(-param.match_radius);
            delta delta_max(+param.match_radius);
            
            // 通过每个区域(bin)的accumulator中的各个偏移值计算出最大偏移值
            if (delta_accu[v_bin*u_bin_num+u_bin].size()>0) {
                
                // init displacements 'delta' to 'infinite'
                delta_min = delta(+1000000);
                delta_max = delta(-1000000);
                
                // find minimum and maximum displacements
                for (vector<Matcher::delta>::iterator it=delta_accu[v_bin*u_bin_num+u_bin].begin();
                    it!=delta_accu[v_bin*u_bin_num+u_bin].end(); it++) {
                    for (int32_t i=0; i<num_stages*2; i++) {
                        if (it->val[i]<delta_min.val[i]) delta_min.val[i] = it->val[i];
                        if (it->val[i]>delta_max.val[i]) delta_max.val[i] = it->val[i];
                    }
                }
            }
        
            // 将最大偏移值进一步处理为搜索范围
            range r;
            for (int32_t i=0; i<num_stages; i++) {
                
                // bound minimum search range to 20x20
                float delta_u = delta_max.val[i*2+0]-delta_min.val[i*2+0];
                if (delta_u<20) {
                    delta_min.val[i*2+0] -= ceil((20-delta_u)/2);
                    delta_max.val[i*2+0] += ceil((20-delta_u)/2);
                }
                float delta_v = delta_max.val[i*2+1]-delta_min.val[i*2+1];
                if (delta_v<20) {
                    delta_min.val[i*2+1] -= ceil((20-delta_v)/2);
                    delta_max.val[i*2+1] += ceil((20-delta_v)/2);
                }
                
                // set range for this bin
                r.u_min[i] = delta_min.val[i*2+0];
                r.u_max[i] = delta_max.val[i*2+0];
                r.v_min[i] = delta_min.val[i*2+1];
                r.v_max[i] = delta_max.val[i*2+1];
            }
            ranges.push_back(r);      
        }
    }
    
    // free bin accumulator memory
    delete []delta_accu;
}
    \end{minted}
    \begin{minted}[linenos, breaklines, breakanywhere, mathescape]{c++}
bool VisualOdometry::updateMotion () {
  
  // 调用estimateMotion预测当前姿态变换
  vector<double> tr_delta = estimateMotion(p_matched);
  
  // on failure
  if (tr_delta.size()!=6)
    return false;
  
  // set transformation matrix (previous to current frame)
  Tr_delta = transformationVectorToMatrix(tr_delta);
  Tr_valid = true;
  
  // success
  return true;
}
    \end{minted}
    \begin{minted}[linenos, breaklines, breakanywhere, mathescape]{c++}
vector<double> VisualOdometryStereo::estimateMotion (vector<Matcher::p_match> p_matched) {

    // compute minimum distance for RANSAC samples
    double width=0,height=0;
    for (vector<Matcher::p_match>::iterator it=p_matched.begin(); it!=p_matched.end(); it++) {
        if (it->u1c>width)  width  = it->u1c;
        if (it->v1c>height) height = it->v1c;
    }
    double min_dist = min(width,height)/3.0;
    
    // get number of matches
    int32_t N  = p_matched.size();
    if (N<6)
        return vector<double>();

    // allocate dynamic memory
    ...

    // project matches of previous image into 3d
    // 把之前一帧的匹配点投影在3D坐标内
    /* 
    $d$代表水平视差, $B$代表baseline(水平基线)
    $d=max(u_l-u_r, 0.0001)$
    $Z=\frac{f\times B}{d}$
    $X=(u-c_u)\frac{B}{d}$
    $Y=(v-c_v)\frac{B}{d}$
    */
    for (int32_t i=0; i<N; i++) {
        double d = max(p_matched[i].u1p - p_matched[i].u2p,0.0001f);
        X[i] = (p_matched[i].u1p-param.calib.cu)*param.base/d;
        Y[i] = (p_matched[i].v1p-param.calib.cv)*param.base/d;
        Z[i] = param.calib.f*param.base/d;
    }

    // loop variables
    vector<double> tr_delta;
    vector<double> tr_delta_curr;
    tr_delta_curr.resize(6);
    
    // clear parameter vector
    inliers.clear();

    // 进行ransac_iters轮RANSAC算法匹配
    for (int32_t k=0;k<param.ransac_iters;k++) {

        // 随机选择3个观测点作为初始inlier
        vector<int32_t> active = getRandomSample(N,3);

        // clear parameter vector
        for (int32_t i=0; i<6; i++)
        tr_delta_curr[i] = 0;

        // minimize reprojection errors
        VisualOdometryStereo::result result = UPDATED;
        int32_t iter=0;
        // 迭代寻找到最优的投影参数
        while (result==UPDATED) {
            result = updateParameters(p_matched,active,tr_delta_curr,1,1e-6);
            if (iter++ > 20 || result==CONVERGED)
                break; //迭代20次或者结果收敛就停止迭代
        }

        // overwrite best parameters if we have more inliers
        if (result!=FAILED) {
            vector<int32_t> inliers_curr = getInlier(p_matched,tr_delta_curr);
            // 获取符合当前模型的内点(inlier)
            if (inliers_curr.size()>inliers.size()) {
                inliers = inliers_curr;
                tr_delta = tr_delta_curr;
                // 模型的优劣以能匹配到的inlier数量决定
                // 保留能匹配到更多inlier的模型
            }
        }
    }
    
    // 最后根据匹配到的inlier修正模型
    if (inliers.size()>=6) {
        int32_t iter=0;
        VisualOdometryStereo::result result = UPDATED;
        while (result==UPDATED) {     
        result = updateParameters(p_matched,inliers,tr_delta,1,1e-8);
        if (iter++ > 100 || result==CONVERGED)
            break;
        }

        // not converged
        if (result!=CONVERGED)
            success = false;

    // not enough inliers
    } else {
        success = false;
    }

    ...
    
    // parameter estimate succeeded?
    if (success) return tr_delta;
    else         return vector<double>();
}
    \end{minted}
    \begin{minted}[linenos, breaklines, breakanywhere, mathescape]{c++}
VisualOdometryStereo::result VisualOdometryStereo::updateParameters(vector<Matcher::p_match> &p_matched,vector<int32_t> &active,vector<double> &tr,double step_size,double eps) {
    
    // we need at least 3 observations
    if (active.size()<3)
        return FAILED;
    
    // extract observations and compute predictions
    computeObservations(p_matched,active); 
    //将active点的“当前帧”坐标放入p_observation中
    computeResidualsAndJacobian(tr,active); 
    // 计算残差与雅克比行列式

    // init
    Matrix A(6,6);
    Matrix B(6,1);

    // fill matrices A and B
    for (int32_t m=0; m<6; m++) {
        for (int32_t n=0; n<6; n++) {
            double a = 0;
            for (int32_t i=0; i<4*(int32_t)active.size(); i++) {
                a += J[i*6+m]*J[i*6+n];
            }
            A.val[m][n] = a;
        }
        double b = 0;
        for (int32_t i=0; i<4*(int32_t)active.size(); i++) {
            b += J[i*6+m]*(p_residual[i]);
        }
        B.val[m][0] = b;
    }

    // perform elimination
    if (B.solve(A)) { //如果模型合理
        bool converged = true;
        for (int32_t m=0; m<6; m++) {
            tr[m] += step_size*B.val[m][0];
            if (fabs(B.val[m][0])>eps)
                converged = false;
                // 如果某项参数更优
        }
        if (converged)
            return CONVERGED;
        else
            return UPDATED;
    } else {
        return FAILED;
    }
}
    \end{minted}
    \begin{minted}[linenos, breaklines, breakanywhere, mathescape]{c++}
vector<int32_t> VisualOdometryStereo::getInlier(vector<Matcher::p_match> &p_matched,vector<double> &tr) {

    // 把所有观测值标记为active
    vector<int32_t> active;
    for (int32_t i=0; i<(int32_t)p_matched.size(); i++)
        active.push_back(i);

    computeObservations(p_matched,active);
    computeResidualsAndJacobian(tr,active);

    vector<int32_t> inliers;
    for (int32_t i=0; i<(int32_t)p_matched.size(); i++)
        if (pow(p_observe[4*i+0]-p_predict[4*i+0],2)+pow(p_observe[4*i+1]-p_predict[4*i+1],2) +
            pow(p_observe[4*i+2]-p_predict[4*i+2],2)+pow(p_observe[4*i+3]-p_predict[4*i+3],2) < param.inlier_threshold*param.inlier_threshold)
        inliers.push_back(i);
        // 如果某个点观测值与预测值的偏差小于阈值,就将其标记为inlier
    return inliers;
}
    \end{minted}
    \begin{minted}[linenos, breaklines, breakanywhere, mathescape]{c++}
Matrix VisualOdometry::transformationVectorToMatrix (vector<double> tr) {

    ...

    // 计算姿态变换矩阵
    Matrix Tr(4,4);
    Tr.val[0][0] = +cy*cz;          Tr.val[0][1] = -cy*sz;          Tr.val[0][2] = +sy;    Tr.val[0][3] = tx;
    Tr.val[1][0] = +sx*sy*cz+cx*sz; Tr.val[1][1] = -sx*sy*sz+cx*cz; Tr.val[1][2] = -sx*cy; Tr.val[1][3] = ty;
    Tr.val[2][0] = -cx*sy*cz+sx*sz; Tr.val[2][1] = +cx*sy*sz+sx*cz; Tr.val[2][2] = +cx*cy; Tr.val[2][3] = tz;
    Tr.val[3][0] = 0;               Tr.val[3][1] = 0;               Tr.val[3][2] = 0;      Tr.val[3][3] = 1;
    return Tr;
}
    \end{minted}
    \section{实验运行结果}
\end{document}